% This is samplepaper.tex, a sample chapter demonstrating the
% LLNCS macro package for Springer Computer Science proceedings;
% Version 2.21 of 2022/01/12
%
\documentclass[runningheads]{llncs}
%
\usepackage[T1]{fontenc}
% T1 fonts will be used to generate the final print and online PDFs,
% so please use T1 fonts in your manuscript whenever possible.
% Other font encondings may result in incorrect characters.
%
\usepackage{graphicx}
% Used for displaying a sample figure. If possible, figure files should
% be included in EPS format.
%
% If you use the hyperref package, please uncomment the following two lines
% to display URLs in blue roman font according to Springer's eBook style:
%\usepackage{color}
%\renewcommand\UrlFont{\color{blue}\rmfamily}
%\urlstyle{rm}
%
\usepackage{listings}
% Used to display Python code.
%
\begin{document}
%
\title{Property-Based Testing of Data Analysis Scripts}
\subtitle{A Focus on Hypothesis for Python}
%
%\titlerunning{Abbreviated paper title}
% If the paper title is too long for the running head, you can set
% an abbreviated paper title here
%
\author{Jean-Sebastian de Wet \and
Jan-Philipp Kiel \and
Pascal Mager}
%
\authorrunning{Group Criterion}
% First names are abbreviated in the running head.
% If there are more than two authors, 'et al.' is used.
%
\institute{University of Cologne, Cologne, Germany}
%
\maketitle              % typeset the header of the contribution
%
\begin{abstract}
This paper explores property-based testing as a method to ensure data analysis scripts' reliability, especially in DLR research using Python, Pandas, and Matplotlib. It outlines challenges with traditional testing in scenarios with diverse data values, emphasizing the need for innovative testing strategies. The paper thoroughly covers property-based testing, including its history, key principles, use cases, and integration in the test pyramid. It then focuses on Hypothesis for Python, a powerful tool for property-based testing, discussing its use, integration with pytest, and unique features. Real-world application is demonstrated with code examples, highlighting how property-based testing, especially with Hypothesis, strengthens data analysis scripts' reliability. The paper concludes by summarizing key findings and emphasizing the crucial role property-based testing, like Hypothesis, plays in boosting researchers' confidence with unknown data.

\keywords{Property-Based Testing \and Data Analysis Scripts \and Hypothesis \and Python \and pytest \and Reliability \and Test Pyramid \and Code Examples \and DLR Research.}
\end{abstract}
%
%
%
% [Page 1]
\section{Introduction}
\begin{itemize}
  \item Briefly introduce the context of data analysis in DLR and the challenges associated with traditional testing approaches.
  \item Highlight the need for property-based testing in scenarios with large possible value ranges.\cite{MacIver2019}
\end{itemize}

% [Page 1-2]
\section{Background}
\begin{itemize}
  \item Discuss the common data analysis tools used by DLR researchers (Python, Pandas, Matplotlib).
  \item Explain the challenges of writing tests for data analysis scripts.
  \item Emphasize the importance of confidence in the face of unknown data.
\end{itemize}

% [Pages 2-5]
\section{Overview of Property-Based Testing}
\subsection{History of Property-Based Testing}
\begin{itemize}
  \item Trace the historical development of property-based testing.
\end{itemize}

\subsection{How Property-Based Testing Works}
\begin{itemize}
  \item Explain the core concept of property-based testing.
  \item Discuss the idea of defining invariants for functions.
\end{itemize}

\subsection{Main Use Cases}
\begin{itemize}
  \item Explore various scenarios where property-based testing is particularly useful.
  \item Discuss how it complements traditional testing approaches.
\end{itemize}

\subsection{Position in the Test Pyramid}
\begin{itemize}
  \item Describe where property-based testing fits in the test pyramid.
\end{itemize}

\subsection{Tools and Programming Languages}
\begin{itemize}
  \item Provide an overview of existing property-based testing tools.
  \item Highlight tools for different programming languages.
\end{itemize}

% [Pages 5-7]
\section{Introduction to Hypothesis for Python}
\subsection{Overview of Hypothesis}
\begin{itemize}
  \item Provide a brief introduction to Hypothesis for Python.
  \item Mention its key features and advantages.
\end{itemize}

\subsection{How to Use Hypothesis}
\begin{itemize}
  \item Write a mini how-to guide on using Hypothesis, including integration with pytest.
  \item Include code snippets for better understanding.
\end{itemize}

% [Pages 7-9]
\section{Main Concepts and Features of Hypothesis}
\subsection{Strategies and Data Generation}
\begin{itemize}
  \item Explain the concept of strategies in Hypothesis for generating test data.
\end{itemize}

\subsection{Property-Based Testing with pytest}
\begin{itemize}
  \item Detail how Hypothesis integrates with pytest.
  \item Provide examples of test functions using Hypothesis.
\end{itemize}

\subsection{Data Analysis Applications and Benefits}
\begin{itemize}
  \item Discuss how data analysis applications can benefit from Hypothesis.
  \item Reference specific features, such as the support for NumPy.
\end{itemize}

% [Pages 9-13]
\section{Application of Hypothesis in Data Analysis}
\subsection{Code Examples}
\begin{itemize}
  \item Provide practical code examples demonstrating the use of Hypothesis in data analysis scripts.
  \item Showcase scenarios where property-based testing adds value.
\end{itemize}

\begin{lstlisting}[language=Python]
    import numpy as np
        
    def incmatrix(genl1,genl2):
        m = len(genl1)
        n = len(genl2)
        M = None #to become the incidence matrix
        VT = np.zeros((n*m,1), int)  #dummy variable
        
        #compute the bitwise xor matrix
        M1 = bitxormatrix(genl1)
        M2 = np.triu(bitxormatrix(genl2),1) 
    
        for i in range(m-1):
            for j in range(i+1, m):
                [r,c] = np.where(M2 == M1[i,j])
                for k in range(len(r)):
                    VT[(i)*n + r[k]] = 1;
                    VT[(i)*n + c[k]] = 1;
                    VT[(j)*n + r[k]] = 1;
                    VT[(j)*n + c[k]] = 1;
                    
                    if M is None:
                        M = np.copy(VT)
                    else:
                        M = np.concatenate((M, VT), 1)
                    
                    VT = np.zeros((n*m,1), int)
        
        return M
\end{lstlisting}

\subsection{Illustrative Cases}
\begin{itemize}
  \item Present specific cases where Hypothesis helped discover issues in data analysis scripts.
\end{itemize}

% [Page 13-14]
\section{Conclusion}
\begin{itemize}
  \item Summarize the key points discussed in the paper.
  \item Emphasize the importance of property-based testing, particularly with tools like Hypothesis, in enhancing the reliability of data analysis scripts.
\end{itemize}
%
% ---- Bibliography ----
%
% BibTeX users should specify bibliography style 'splncs04'.
% References will then be sorted and formatted in the correct style.
%
\bibliographystyle{splncs04}
\bibliography{references}
\end{document}
